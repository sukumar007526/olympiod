\documentclass[12pt,-letter paper]{article}
\usepackage{siunitx}
\usepackage{setspace}
\usepackage{gensymb}
\usepackage{xcolor}
\usepackage{caption}
%\usepackage{subcaption}
\doublespacing
\singlespacing
\usepackage[none]{hyphenat}
\usepackage{amssymb}
\usepackage{relsize}
\usepackage[cmex10]{amsmath}
\usepackage{mathtools}
\usepackage{amsmath}
\usepackage{commath}
\usepackage{amsthm}
\interdisplaylinepenalty=2500
%\savesymbol{iint}
\usepackage{txfonts}
%\restoresymbol{TXF}{iint}
\usepackage{wasysym}
\usepackage{amsthm}
\usepackage{mathrsfs}
\usepackage{txfonts}
\let\vec\mathbf{}
\usepackage{stfloats}
\usepackage{float}
\usepackage{cite}
\usepackage{cases}
\usepackage{subfig}
%\usepackage{xtab}
\usepackage{longtable}
\usepackage{multirow}
%\usepackage{algorithm}
\usepackage{amssymb}
%\usepackage{algpseudocode}
\usepackage{enumitem}
\usepackage{mathtools}
%\usepackage{eenrc}
%\usepackage[framemethod=tikz]{mdframed}
\usepackage{listings}
%\usepackage{listings}
\usepackage[latin1]{inputenc}
%%\usepackage{color}{   
%%\usepackage{lscape}
\usepackage{textcomp}
\usepackage{titling}
\usepackage{hyperref}
%\usepackage{fulbigskip}   
\usepackage{tikz}
\usepackage{graphicx}
\lstset{
  frame=single,
  breaklines=true
}
\let\vec\mathbf{}
\usepackage{enumitem}
\usepackage{graphicx}
\usepackage{siunitx}
\let\vec\mathbf{}
\usepackage{enumitem}
\usepackage{graphicx}
\usepackage{enumitem}
\usepackage{tfrupee}
\usepackage{amsmath}
\usepackage{amssymb}
\usepackage{mwe} % for blindtext and example-image-a in example
\usepackage{wrapfig}
\graphicspath{{figs/}}
\providecommand{\mydet}[1]{\ensuremath{\begin{vmatrix}#1\end{vmatrix}}}
\providecommand{\myvec}[1]{\ensuremath{\begin{bmatrix}#1\end{bmatrix}}}
\providecommand{\cbrak}[1]{\ensuremath{\left\{#1\right\}}}
\providecommand{\brak}[1]{\ensuremath{\left(#1\right)}}
\begin{document}
\begin{enumerate}
 
	\item Consider two coplanar circles of radii $R$ and $r$ $\brak{R > r}$ with the same center. Let $P$ be a fixed point on the smaller circle and $B$ a variable point on the larger circle. The line $BP$ meets the larger circle again at $C$. The perpendicular $l$ to $BP$ at $P$ meets the smaller circle again at $A$. (If $l$ is tangent to the circle at $P$ then $A = P$)\\
		$\brak{i}$ Find the set of values of $BC^2+CA^2+AB^2$ \\
		$\brak{ii}$ Find the locus of the midpoint of $BC$.
		
	\item Let $n$ be a positive integer and let $A_1, A_2, \dots, A_{2n+1}$ be subsets of a set $B$. Suppose that \\
		$\brak{a}$ Each $A_i$ has exactly $2n$ elements,\\ 
		$\brak{b}$ Each $A_i \cap A_j \brak{1\leq i \leq j\leq 2n+1}$contains exactly one element, and \\
		$\brak{c}$ Every element of $B$ belongs to at least two of the $A_i$.\\
		\\
		For which values of $n$ can one assign to every element of $B$ one of the numbers $0$ and $1$ in such a way that $A_i$ has $0$ assigned to exactly $n$ of its elements?

	\item A function $f$ is defined on the positive integers by\\ 
		\begin{align*}	
f\brak{1}=1, f\brak{3}=3, \\
f\brak{2n}=f\brak{n}, \\
		f\brak{4n+1}=2f\brak{2n+1}-f\brak{n},\\
		f\brak{4n+3}=3f\brak{2n+1}-2f\brak{n},\\ \end{align*}
		for all positive integers n.\\
		Determine the number of positive integers $n$, less than or equal to $1988$, for which $f(n) = n$.

	\item Show that set of real numbers x which satisfy the inequality \\
		\begin{align*}\sum{k=1}^{70}\frac{k}{x-k}\geq \frac{5}{4}\\ \end{align*}
		is a union of disjoint intervals, the sum of whose lengths is $1988$

	\item $ABC$ is a triangle right-angled at $A$, and $D$ is the foot of the altitude from $A$. The straight line joining the incenters of the triangles $ABD$, $ACD$ intersects the sides $AB$, $AC$ at the points $K$, $L$ respectively. $S$ and $T$ denote the areas of the triangles $ABC$ and $AKL$ respectively. Show that $S\geq 2T$.

	\item Let $a$ and $b$ be positive integers such that $ab + 1$ divides $a^2 + b^2$. Show that \\
		\begin{align*} \frac{a^2+b^2}{ab+1} \end{align*}\\
			is the square of an integer.
\end{enumerate}
\end{document}
