\documentclass[12pt,-letter paper]{article}
\usepackage{siunitx}
\usepackage{setspace}
\usepackage{gensymb}
\usepackage{xcolor}
\usepackage{caption}
%\usepackage{subcaption}
\doublespacing
\singlespacing
\usepackage[none]{hyphenat}
\usepackage{amssymb}
\usepackage{relsize}
\usepackage[cmex10]{amsmath}
\usepackage{mathtools}
\usepackage{amsmath}
\usepackage{commath}
\usepackage{amsthm}
\interdisplaylinepenalty=2500
%\savesymbol{iint}
\usepackage{txfonts}
%\restoresymbol{TXF}{iint}
\usepackage{wasysym}
\usepackage{amsthm}
\usepackage{mathrsfs}
\usepackage{txfonts}
\let\vec\mathbf{}
\usepackage{stfloats}
\usepackage{float}
\usepackage{cite}
\usepackage{cases}
\usepackage{subfig}
%\usepackage{xtab}
\usepackage{longtable}
\usepackage{multirow}
%\usepackage{algorithm}
\usepackage{amssymb}
%\usepackage{algpseudocode}
\usepackage{enumitem}
\usepackage{mathtools}
%\usepackage{eenrc}
%\usepackage[framemethod=tikz]{mdframed}
\usepackage{listings}
%\usepackage{listings}
\usepackage[latin1]{inputenc}
%%\usepackage{color}{   
%%\usepackage{lscape}
\usepackage{textcomp}
\usepackage{titling}
\usepackage{hyperref}
%\usepackage{fulbigskip}   
\usepackage{tikz}
\usepackage{graphicx}
\lstset{
  frame=single,
    breaklines=true
    }
    \let\vec\mathbf{}
    \usepackage{enumitem}
    \usepackage{graphicx}
    \usepackage{siunitx}
    \let\vec\mathbf{}
    \usepackage{enumitem}
    \usepackage{graphicx}
    \usepackage{enumitem}
    \usepackage{tfrupee}
    \usepackage{amsmath}
    \usepackage{amssymb}
    \usepackage{mwe} % for blindtext and example-image-a in example
    \usepackage{wrapfig}
    \graphicspath{{figs/}}
    \providecommand{\mydet}[1]{\ensuremath{\begin{vmatrix}#1\end{vmatrix}}}
    \providecommand{\myvec}[1]{\ensuremath{\begin{bmatrix}#1\end{bmatrix}}}
    \providecommand{\cbrak}[1]{\ensuremath{\left\{#1\right\}}}
    \providecommand{\brak}[1]{\ensuremath{\left(#1\right)}}
    \begin{document}
   \begin{enumerate}
  
	   \item Let $p_n \brak{k}$ be the number of permutations of the set $\cbrak{1,\dots,n}$, $n\geq1$,which have exactly $k$ fixed points.Prove that \\
	   \begin{align*}   \sum_{k=0}^{n} k \cdot p_n\brak{k} = n 
	   \end{align*} 
		   (Remark:A permtation $f$ of a set $S$ is one-to-one mapping of $S$ onto itself.An element $i$ in $S$ is called a fixed point of the the permutation $f$ if f\brak{i}=i. )
   
	   \item In an acute-angled triangle $ABC$ the interior bisector of the angle $A$ intersects $BC$ at $L$ and intersects the circumcircle of $ABC$ again at $N$. From point $L$ perpendiculars are drawn to $AB$ and $AC$, the feet of these perpendiculars being $K$ and $M$ respectively. Prove that the quadrilateral $AKNM$ and the triangle $ABC$ have equal areas.
	 
	   \item Let $x_1, x_2, \dots, x_n$ be real numbers satisfying $x^2_1 +x^2_2 +\dots +x^2_n = 1$. Prove that for every integer $k\geq2$ there are integers $a_1, a_2, \dots , a_n$, not all $0$, such that $\mydet{a_i}\leq k-1$ for all $i$ and \\
		   \begin{align*} \mydet{a_1x_1+a_2x_2+\dots+a_nx_n} \leq \frac{\brak{k-1}\sqrt{n}}{k^n-1} \end{align*}
		   
           \item Prove that there is no function $f$ from the set of non-negative integers into itself such that $f\brak{f\brak{n}}=n+1987$ for every $n$.

	  \item Let $n$ be an integer greater than or equal to $3$. Prove that there is a set of $n$ points in the plane such that the distance between any two points is irrational and each set of three points determines a non-degenerate triangle with rational area.


	  \item Let $n$ be an integer greater than or equal to $2$. Prove that if $k^2+k+n$ is prime for all integers $k$ such that $0\leq k\leq \sqrt{n/3}$, then $k^2+k+n$ is prime for all integers $k$ such that $0\leq k\leq n-2$ 
		



   \end{enumerate}
   \end{document}
