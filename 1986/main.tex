\documentclass[12pt,-letter paper]{article}
\usepackage{siunitx}
\usepackage{setspace}
\usepackage{gensymb}
\usepackage{xcolor}
\usepackage{caption}
%\usepackage{subcaption}
\doublespacing
\singlespacing
\usepackage[none]{hyphenat}
\usepackage{amssymb}
\usepackage{relsize}
\usepackage[cmex10]{amsmath}
\usepackage{mathtools}
\usepackage{amsmath}
\usepackage{commath}
\usepackage{amsthm}
\interdisplaylinepenalty=2500
%\savesymbol{iint}
\usepackage{txfonts}
%\restoresymbol{TXF}{iint}
\usepackage{wasysym}
\usepackage{amsthm}
\usepackage{mathrsfs}
\usepackage{txfonts}
\let\vec\mathbf{}
\usepackage{stfloats}
\usepackage{float}
\usepackage{cite}
\usepackage{cases}
\usepackage{subfig}
%\usepackage{xtab}
\usepackage{longtable}
\usepackage{multirow}
%\usepackage{algorithm}
\usepackage{amssymb}
%\usepackage{algpseudocode}
\usepackage{enumitem}
\usepackage{mathtools}
%\usepackage{eenrc}
%\usepackage[framemethod=tikz]{mdframed}
\usepackage{listings}
%\usepackage{listings}
\usepackage[latin1]{inputenc}
%%\usepackage{color}{   
%%\usepackage{lscape}
\usepackage{textcomp}
\usepackage{titling}
\usepackage{hyperref}
%\usepackage{fulbigskip}   
\usepackage{tikz}
\usepackage{graphicx}
\lstset{
  frame=single,
  breaklines=true
}
\let\vec\mathbf{}
\usepackage{enumitem}
\usepackage{graphicx}
\usepackage{siunitx}
\let\vec\mathbf{}
\usepackage{enumitem}
\usepackage{graphicx}
\usepackage{enumitem}
\usepackage{tfrupee}
\usepackage{amsmath}
\usepackage{amssymb}
\usepackage{mwe} % for blindtext and example-image-a in example
\usepackage{wrapfig}
\graphicspath{{figs/}}
\providecommand{\mydet}[1]{\ensuremath{\begin{vmatrix}#1\end{vmatrix}}}
\providecommand{\myvec}[1]{\ensuremath{\begin{bmatrix}#1\end{bmatrix}}}
\providecommand{\cbrak}[1]{\ensuremath{\left\{#1\right\}}}
\providecommand{\brak}[1]{\ensuremath{\left(#1\right)}}
\begin{document}
\begin{enumerate}

	\item  Let $d$ be any positive integer not equal to $2$, $5$, or $13$. Show that one can find distinct $a$, $b$ in the set $\cbrak{2, 5, 13. d}$ such that $ab-1$ is not a perfect square.
	\item A triangle $A_1A_2A_3$ and a point $P_0 $are given in the plane.We define $A_s=A_s-3$ for all $s\geq4$ .We construct a set of points $P_1$, $P_2$,$P_3$\dots,such that $P_{k+1}$ is the image of $P_k$ under a rotation with center $A_{k+1}$ through angle $120^\circ$ clockwise $\brak{for \space k=0,1,2,3\dots}$ .Prove that if $P_{1986}$=$P_0$, then the triangle $A_1A_2A_3$ is equilateral .
	\item  To each vertex of a regular pentagon an integer is assigned in such a way that the sum of all five numbers is positive. If three consecutive vertices are assigned the numbers $x$,$y$,$z$ respectively and $y<0$ then the following operation is allowed: the numbers $x$,$y$,$z$ are replaced by $x+y$,$-y$,$z+y$ respectively. Such an operation is performed repeatedly as long as at least one of the five numbers is negative. Determine whether this procedure necessarily comes to and end after a finite number of steps.
	\item Let $A$, $B$ be adjacent vertices of a regular n-gon $\brak{n\leq5}$ in the plane having center at $O$. A triangle $XYZ$, which is congruent to and initially conincides with $OAB$, moves in the plane in such a way that $Y$ and $Z$ each trace out the whole boundary of the polygon, $X$ remaining inside the polygon. Find the locus of $X$.
	\item Find all functions $f$, defined on the non-negative real numbers and taking nonnegative real values, such that: \\ 
		$\brak{i}$ $f\brak{xf\brak{y}}f\brak{y}=f\brak{x+y}$ for all $x$,$y\geq0$, \\
		$\brak{ii}$ $f\brak{2}=0$\\
		$\brak{iii}$ $f\brak{x}\neq0$ for $o\leq x <2$
	\item One is given a finite set of points in the plane, each point having integer coordinates. Is it always possible to color some of the points in the set red and the remaining points white in such a way that for any straight line $L$ parallel to either one of the coordinate axes the difference (in absolute value) between the numbers of white point and red points on $L$ is not greater than $1$?

\end{enumerate}
\end{document}
